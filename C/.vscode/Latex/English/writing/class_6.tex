Directions: For this part, you are allowed 30 minutes to write an essay that begins with the 
sentence ``Nowadays more and more people take delight in offering help to the needy." You 
can make comments, cite examples, or use your personal experiences to develop your essay. 
You should write at least 150 words but no more than 200 words.
~\\


给出观点$\Rightarrow$引出一个相关现象,把抽象的观点具象化
$\Rightarrow$举例$\Rightarrow$给出正确的定义$\Rightarrow$总结
~\\


Nowadays more and more people take delight in offering help to the needy. 
(不废话的肯定)In a world marred(v. 破坏,糟蹋) 
by individualism and materialism, seeing people 
extend a helping hand to those in less fortunate 
circumstances is indeed \textbf{
    a scene  that we all yearn(v. 渴望) to witness. (...是我们渴望看到的,引出话题)
}
However, assisting others is not without its challenges and costs. 
Everyone has their own concerns and limitations, 
and our efforts should be to encourage and support 
rather judge or criticize.
~\\


界定道德绑架这个概念:通过质疑别人的道德,强迫别人做...
~\\

(解释概念$\Rightarrow$举例,这样做的不好之处)


Now, the phenomenon of ``moral hijacking(hijack v. 绑架)" has become 
increasingly prevalent. This concept refers to 
the practice of compelling individuals or 
organizations to support a cause by questioning 
their morality or ethics, often leveraging(v. 利用) 
social media as a platform for exerting pressure. 
During fundraising(fundraise v. 募捐) certain campaigns, people 
might feel coerced(coerce v. 强制,威胁) into donating 
because failure to do so could result in being 
labeled as indifferent or selfish by the community. 
Such tactics not only create an atmosphere of guilt 
and obligation but can also lead to resentment 
and a decrease in genuine acts of charity. 


In fact, offering help, whether it be(后面既有单数又有复数可以直接用be) through 
volunteering, financial donations, or simply 
lending an ear, often demands time, energy 
and resources. For some, these are readily 
available, but for others, there may be significant 
personal or financial constraints that make 
such contributions difficult. It's important 
to recognize that every little act of kindness 
counts, regardless of its scale.


\textbf{Not all of us can do great things. But we can do 
small things with great love.(用心做事、哪怕渺小也能做大事)} 
By valuing each act of kindness, irrespective 
of size(irrespective adj. 不考虑 irrespective of sth 不考虑...), we cultivate a more compassionate and 
understanding society, one that cherishes 
the spirit of giving while respecting individual 
boundaries and capacities.