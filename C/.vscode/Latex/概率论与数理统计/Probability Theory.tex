\documentclass[12pt, a4paper, oneside]{ctexbook}
\usepackage{amsmath, amsthm, amssymb, bm, graphicx, hyperref, mathrsfs,diagbox}

\title{{\Huge{\textbf{概率论与数理统计}}}\\——副标题}
\author{Dylaaan}
\date{\today}
\linespread{1.5}
\newtheorem{theorem}{定理}[section]
\newtheorem{definition}[theorem]{定义}
\newtheorem{lemma}[theorem]{引理}
\newtheorem{corollary}[theorem]{推论}
\newtheorem{example}[theorem]{例}
\newtheorem{proposition}[theorem]{命题}

\begin{document}

\maketitle

\pagenumbering{roman}
\setcounter{page}{1}

\begin{center}
    \Huge\textbf{前言}
\end{center}~\

这是笔记的前言部分. 
~\\
\begin{flushright}
    \begin{tabular}{c}
        Dylaaan\\
        \today
    \end{tabular}
\end{flushright}

\newpage
\pagenumbering{Roman}
\setcounter{page}{1}
\tableofcontents
\newpage
\setcounter{page}{1}
\pagenumbering{arabic}
\chapter{概率论的基本概念}

\chapter{随机变量及其分布}
\section{随机变量}
\section{离散型随机变量及其分布}
\section{随机变量的分布函数}
\section{连续性随机变量及其密度函数}
\section{随机变量函数的分布}

\chapter{多元随机变量及其分布}
\section{二元离散型随机变量}
联合概率分布律的性质:
\begin{itemize}
    \item [1.]$p_{ij} \geq 0$ , i,j = 1,2, ...
    \item [2.]$\sum\limits _{i=1}^{\infty}\sum\limits_{j=1}^{\infty}p_{ij}=1$
\end{itemize}

~\\
例1.1设随机变量X在1、2、3、4四个整数中等可能地取一个值,
另一个随机变量Y在1 $\sim$ X中等可能地曲一整数值,
试求(X,Y)的联合概率分布。


(X,Y)的联合概率分布为:
\begin{center}
    \begin{tabular}{|c|c|c|c|c|}
        \hline
        \diagbox{X}{Y}&1&2&3&4\\
        \hline
        1&1/4&0&0&0\\
        \hline
        2&1/8&1/8&0&0\\
        \hline
        3&1/12&1/12&1/12&0\\
        \hline
        4&1/16&1/16&1/16&1/16\\
        \hline
    \end{tabular}
\end{center}

~\\
~\\
~\\
(二)边际分布


对于离散型随机变量(X,Y),联合分布律为
$$P(X=x_i,Y=y_j)=p_{ij},i,j=1,2,...$$


X,Y的边际(边缘)分布律为:


$$P(Y=y_j)=P(X<+\infty,Y=y_j)=\sum _{i=1}^\infty p_{ij}=p_{\cdot j} \text{ j=1,2,...}$$
$$P(X=x_i)=P(X=x_i,Y<+\infty)=\sum _{j=1}^\infty p_{ij}=p_{i\cdot}\text{ i=1,2,...}$$




\newpage
\section{二元随机变量的分布函数}
\textbf{(一)联合分布函数}


定义:设(X,Y)是二元随机变量,对于任意实数x,y,二元
函数$F(x,y)=P\{(X\leq x)\cap (Y\leq y)\}=P(X\leq x,Y\leq y)$
称为二元随机变量(X,Y)的联合分布函数。


~\\
\begin{center}
    \textbf{分布函数F(X,Y)的性质}
\end{center}
\begin{itemize}
    \item [1°]F(x,y)关于x,y单调不减,即:
    $$x_1 < x_2\Rightarrow F(x_1,y) \leq F(x_2,y)$$
    $$y_1 < y_2\Rightarrow F(x,y_1) \leq F(x,y_2)$$
    \item [2°]$0\leq F(x,y)\leq 1$,
    

    对任意x,y,$F(-\infty,y)=F(x,-\infty)=F(-\infty,-\infty)=0$
    \item [3°]F(x,y)关于x,y右连续,即:
    $$\lim\limits_{\epsilon \to 0^+}F(x+\epsilon,y)=F(x,y)$$
    $$\lim\limits_{\epsilon \to 0^+}F(x,y+\epsilon)=F(x,y)$$
    \item [4°]若$x_1<x_2$,$y_1<y_2 \Rightarrow 
    F(x_2,y_2)-F(x_2,y_1)-F(x_1,y_2)+F(x_1,y_1)\geq 0$
\end{itemize}

\textbf{(二)边际(边缘)分布函数}


二元随机变量(X,Y)作为整体,有分布函数F(x,y).X和Y都是
随机变量,它们的分布函数分别记为$F_X(x)$和
$F_Y(y)$,称为边际分布函数.
$$F_X(x)=F(x,+\infty)$$
$$F_Y(y)=F(+\infty,y)$$


\newpage
\section{二元连续性随机变量}
\textbf{(一)联合概率密度函数}


定义:对于二元随机变量$(X,Y)$的分布函数$F(x,y)$,如果存在非负函数
$f$使对于任意$x$,$y$,有$$F(x,y)=\int _{-\infty}^y\int _{-\infty}^x
f(u,v)dudv$$
称$(X,Y)$为二元连续型随机变量,称$f(x,y)$为二元随机变量$(X,Y)$的
(联合)概率密度函数


\textbf{联合密度函数性质}
\begin{itemize}
    \item [1.]$f(x,y)\geq 0$
    \item [2.]$\int _{-\infty}^{+\infty}\int _{-\infty}^{+\infty}
    f(x,y)dxdy=1$
    \item [3.]设$G$是平面上区域,$(X,Y)$落在$G$内的概率
    $P\{(X,Y)\in G\}=\iint \limits_{G}f(x,y)dxdy$
    \item [4.]在$f(x,y)$的连续点$(x,y)$,有
    $\dfrac{\partial ^2 F(x,y)}{\partial x \partial y}=f(x,y)$
\end{itemize}


\textbf{(二)边际(边缘)概率密度函数}


设连续性随机变量$(X,Y)$的密度函数为$f(x,y)$,则$X,Y$的边际概率密度函数为别为:
$$f_X(x)=\int _{-\infty}^{+\infty}f(x,y)dy$$
$$f_Y(y)=\int _{-\infty}^{+\infty}f(x,y)dx$$


\textbf{(三)条件分布函数}


定义:若$P(Y=y)>0$,则在$\{Y=y\}$条件下,$X$的条件分布函数为:
$$F_{X|Y}(x|y)=P(X\leq x|Y=y)=\dfrac{P(X\leq x,Y=y)}{P(Y=y)}$$
若$P(Y=y)=0$,但对任给$\epsilon >0,P(y<Y+\epsilon \leq y+\epsilon)>0$,
则在$\{Y=y\}$条件下,$X$的条件分布函数为:
$$F_{X|Y}(x|y)=\lim\limits_{\epsilon \to 0^+}P(X\leq x|y<Y\leq y+\epsilon)
=\lim\limits_{\epsilon \to 0^+} \dfrac{P(X\leq x,y<Y\leq y+\epsilon)}{P(y<Y\leq y+\epsilon)}$$
仍记为$P(X\leq x|Y=y)$


事实上,
$$F_X(x)=F(x,+\infty)=\int _{-\infty}^x [\int _{-\infty}^{+\infty}f(u,v)dv]du
=\int _{-\infty}^xf_X(u)du$$


同理:
$$F_Y(y)=F(+\infty,y)=\int _{-\infty}^y[\int _{-\infty}^{+\infty}f(u,v)du]dv
=\int _{-\infty}^y f_Y(v)dv$$


定义:条件概率密度函数


设二元随机变量$(X,Y)$的密度函数为$f(x,y)$,$X,Y$的边际密度函数为$f_X(x),f_Y(y)$,
则在$\{Y=y\}$条件下$X$的条件密度函数为:
$$f_{X|Y}(x|y)=\dfrac{f(x,y)}{f_Y(y)},f_Y(y)>0$$


在$\{X=x\}$条件下,$Y$的条件密度函数为:
$$f_{Y|X}(y|x)=\dfrac{f(x,y)}{f_X(x)},f_X(x)>0$$


即$F_{X|Y}(x|y)=\int _{-\infty}^x\dfrac{f(u,y)}{f_Y(y)}du$
\begin{align}
    \because F_{X|Y}(x|y)&=\lim \limits_{\Delta y \to 0^+} 
    \dfrac{P(X\leq x,y<Y\leq y+\Delta y)}{P(y<Y\leq y+\Delta y)}\notag
    \\&= \lim \limits_{\Delta y \to 0^+} 
    \dfrac{\dfrac{1}{\Delta y}\int _{-\infty}^x ds \int _y^{y+\Delta y}f(u,v)dv}{\dfrac{1}{\Delta y}\int _y^{y+\Delta y}f_Y(t)dt}\notag
    \\&=\dfrac{\int _{-\infty}^xf(u,y)du}{f_Y(y)}\notag
    \\&=\int _{-\infty}^x \dfrac{f(u,y)}{f_Y(y)}du\notag
    \\\therefore F_{X|Y}(x|y)=\int _{-\infty}^x\dfrac{f(u,y)}{f_Y(y)}du \notag
\end{align}


\textbf{(四)二元均匀分布与二元正太分布}


二元均匀分布:


若二元随机变量$(X,Y)$在二维有界区域$D$上取值,且具有概率密度函数
$$
f(x,y) = \begin{cases}
    \dfrac{1}{\text{D的面积}}, & (x,y) \in D \\
    0, & \text{其他}
\end{cases}
$$


若$D_1$是$D$的子集,则
$$P\{(X,Y)\in D_1\}=\iint \limits_{D_1}f(x,y)dxdy
=\dfrac{D_1\text{的面积}}{D\text{的面积}}$$


二元正太分布:


设二元随机变量$(X,Y)$的概率密度函数为:
$$f(x,y)=\dfrac{1}{2\pi \sigma_1\sigma_2 \sqrt{1-\rho^2}}
\cdot exp\{\dfrac{-1}{2(1-\rho^2)}[\dfrac{(x-\mu_1)^2}{\sigma_1^2}
-2\rho\dfrac{(x-\mu_1)(y-\mu_2)}{\sigma_1\sigma_2}+
\dfrac{(y-\mu_2)^2}{\sigma_2^2}]\}$$


$(-\infty<x<+\infty,-\infty<y<+\infty)$


其中$\mu_1,\mu_2,\sigma_1,\sigma_2$都是常数,
且$\sigma_1>0,\sigma_2>0,-1<\rho<1$


称$(X,Y)$为服从参数为$\mu_1,\mu_2,\sigma_1,\sigma_2,\rho$
的二元正太分布,记为:
$$(X,Y)\sim N(\mu_1,\mu_2,\sigma_1^2,\sigma_2^2,\rho)$$


\begin{align}
    f_{Y|X}(y|x)&=\dfrac{f(x,y)}{f_X(x)}\notag
    \\&=\dfrac{1}{\sqrt{2\pi}\sigma_2\sqrt{1-\rho^2}}
    \cdot exp\{\dfrac{-1}{2(1-\rho^2)\sigma_2^2}[y-(\mu_2+\rho\dfrac{\sigma_2}{\sigma_1}(x-\mu_1))]^2\}\notag
\end{align}


即在$\{X=x\}$条件下,$Y$的条件分布是正太分布
$$N(\mu_2+\rho\dfrac{\sigma_2}{\sigma_1}(x-\mu_1),(1-\rho^2)\sigma_2^2)$$


同理,在$\{Y=y\}$条件下,$X$的条件分布是正太分布
$$N(\mu_1+\rho\dfrac{\sigma_1}{\sigma_2}(y-\mu_2),(1-\rho^2)\sigma_1^2)$$


\newpage
\section{随机变量的独立性}
定义:设$F(x,y)$及$F_X(x),F_Y(y)$分别是随机变量
$(X,Y)$的联合分布函数及边际分布函数,
若对所有实数$x,y$有
$$P(X\leq x,Y\leq y)=P(X\leq x)P(Y\leq y)$$

即:
$$F(x,y)=F_X(x)F_Y(y)$$


称随机变量$X,Y$相互独立


若$(X,Y)$是离散型随机变量,则$X,Y$相互独立的条件
等价于:$P(X=x_i,Y=y_j)$,即:
$p_{ij}=p_{i\cdot}p_{\cdot j}$对一切$i,j$都成立.


若$(X,Y)$是连续型随机变量,$f(x,y),f_X(x),f_Y(y)$
分别是$(X,Y)$的联合密度函数和边缘密度函数,则$X,Y$
相互独立的条件等价于:$f(x,y)=f_X(x)f_Y(y)$几乎处处成立;
即平面上除去零“面积”集外,处处成立

$X$、$Y$独立$\iff\forall x,y,F(x,y)=F_X(x)F_Y(y)$

$X$、$Y$离散:$\iff P(X=x_i,Y=y_j)=P(X=x_i)P(Y=y_j)$

$\iff P(Y=y_j|X=x_i)=P(Y=y_j)$

$X$、$Y$连续:$\iff f(x,y)=f_X(x)f_Y(y)$

$\iff f_{Y|X}(y|x)=f_Y(y),\forall x,y$
~\\

定理3.4.1:连续型随机变量$X,Y$相互独立的充分必要条件是:
$$f(x,y)=m(x)\cdot n(y),|x|<+\infty,|y|<+\infty$$

即:\begin{itemize}
    \item [1.]$f$可以拆成分别关于$x,y$的两个函数的积
    \item [2.]$x,y$积分区域没有关联
\end{itemize}
~\\

例4.5 证明:对于二维正态随机变量$(X,Y)$,$X$与$Y$相互独立的充要条件是参数$\rho=0$

因为$(X,Y)$的概率密度函数为:
$$f(x,y)=\dfrac{1}{2\pi\sigma_1\sigma_2\sqrt{1-\rho^2}}\cdot
exp\{\dfrac{-1}{2(1-\rho^2)}[\dfrac{(x-\mu_1)^2}{\sigma_1^2}-
2\rho\dfrac{(x-\mu_1)(y-\mu_2)}{\sigma_1\sigma_2}+
\dfrac{(y-\mu_2)^2}{\sigma_2^2}]\}$$
又其边际密度函数的乘积为:
$$f_X(x)f_Y(y)=\dfrac{1}{2\pi\sigma_1\sigma_2}\cdot
exp\{-\dfrac{1}{2}[\dfrac{(x-\mu_1)^2}{\sigma_1^2}+
\dfrac{(y-\mu_2)^2}{\sigma_2^2}]\}$$

$\Leftarrow$:如果$\rho=0$,则对$\forall x,y$,有
$f(x,y)=f_X(x)f_Y(y)$,即$X,Y$相互独立

$\Rightarrow$:反之,若$X,Y$相互独立,由于$f(x,y),f_X(x),f_Y(y)$都是连续函数,

故对$\forall x,y,$有$f(x,y)=f_X(x)f_Y(y)$

特别的,有$f(\mu_1,\mu_2)=f_X(\mu_1)f_Y(\mu_2),$

即:$\dfrac{1}{2\pi\sigma_1\sigma_2\sqrt{1-\rho^2}}=\dfrac{1}{2\pi\sigma_1\sigma_2}$


\begin{center}
    \textbf{一般n元随机变量的一些概念和结果}
\end{center}

n元随机变量定义:

设$E$是一个随机试验,它的样本空间是$S=\{e\};$
设$X_1=X_1(e),X_2=X_2(e),...,X_n=X_n(e)$是定义在$S$上的随机变量,
由它们构成的一个$n$元向量$(X_1,X_2,...X_n)$称为$n$元随机变量
~\\

n元随机变量分布函数:

对于任意$n$个实数$x_1,x_2,...,x_n,$n元函数:

$F(x_1,x_2,...,x_n)=P(X_1\leq x_1,X_2\leq x_2,...,X_n\leq x_n)$

称为n元随机变量$(X_1,X_2,...,X_n)$的分布函数
~\\

离散型随机变量的分布律、连续型随机变量的概率密度函数与之前的定义相同。
~\\

\begin{center}
    \textbf{多元随机变量相互独立:}
\end{center}


若对于所有的$x_1,x_2,...,x_n,$有:
$$F(x_1,x_2,...,x_n)=F_{X_1}(x_1)F_{X_2}(x_2)...F_{X_n}(x_n)$$

则称$X_1,X_2,...,X_n$是相互独立的
~\\

$(X_1,X_2,...,X_m)$与$(Y_1,Y_2,...,Y_n)$的独立性


设$(X_1,X_2,...,X_m)$的分布函数为$F_1(x_1,x_2,...,x_m),$
$(Y_1,Y_2,...,Y_n)$的分布函数为$F_2(y_1,y_2,...,y_n),$
$(X_1,X_2,...,X_n,Y_1,Y_2,...,Y_n)$的分布函数为:
$$F(x_1,x_2,...,x_m,y_1,y_2,...,y_n)$$

若$F(x_1,x_2,...,x_m,y_1,y_2,...,y_n)=F_1(x_1,x_2,...,x_n)F_2(y_1,y_2,...,y_n),$

称$(X_1,X_2,...,X_m)$与$(Y_1,Y_2,...,Y_n)$相互独立
~\\

定理:设$(X_1,X_2,...,X_m)$与$(Y_1,Y_2,...,Y_n)$相互独立,
则$X_i(i=1,2,...,m)$与$Y_j(j=1,2,...,n)$相互独立


若$h(x_1,x_2,...,x_m)$与$g(y_1,y_2,...,y_n)$是连续函数,
则$h(X_1,X_2,...,X_m)$和$g(Y_1,Y_2,...,Y_n)$相互独立
~\\

\begin{center}
    \textbf{(一)$Z=X+Y$的分布}
\end{center}
设$(X,Y)$为离散型随机变量,分布律为$P(X=x_i,Y=y_j)=p_{ij},i,j=1,2,...$
设$Z$的可能取值为$z_1,z_2,...,z_k,...,$则$Z=X+Y$的分布律为
$$P(Z=z_k)=P(X+Y=z_k)=\sum\limits_{i=1}^{+\infty}P(X=x_i,Y=z_k-x_i),k=1,2,...$$
或$P(Z=z_k)=\sum\limits_{j=1}^{+\infty}P(X=z_k-y_j,Y=y_j),k=1,2,...$

特别的,当$X$与$Y$相互独立时,
$$P(Z=z_k)=\sum_{i=1}^{+\infty}P(X=x_i)P(Y=z_k-x_i),k=1,2,...$$
或
$$P(Z=z_k)=\sum_{j=1}^{+\infty}P(X=z_k-y_j)P(Y=y_j),k=1,2,...$$
~\\


设连续型随机变量$(X,Y)$的密度函数为$f(x,y)$,则$Z=X+Y$
的分布函数为:
\begin{align}
    F_Z(z)&=P(Z\leq z)=\iint\limits_{x+y\leq z}f(x,y)dxdy\notag
    \\&=\int_{-\infty}^{+\infty}[\int_{-\infty}^{z-y}f(x,y)dx]dy\notag
    \\&=\int_{-\infty}^{+\infty}[\int_{-\infty}^zf(u-y,y)du]dy\notag
    \\&=\int_{-\infty}^z[\int_{-\infty}^{+\infty}f(u-y,y)dy]du\notag
    \\&=\int_{-\infty}^zf_Z(u)du\notag
\end{align}
固定$z,y,$令$u=x+y$

故$Z$的密度函数为:
$$f_Z(z)=\int_{-\infty}^{+\infty}f(z-y,y)dy$$

由$X,Y$的对称性:
$$f_Z(z)=\int_{-\infty}^{+\infty}f(x,z-x)dx$$

当$X$与$Y$相互独立时,
$$f_Z(z)=\int_{-\infty}^{+\infty}f_X(z-y)f_Y(y)dy=
\int_{-\infty}^{+\infty}f_X(x)f_Y(z-x)dx$$
称为卷积公式
\begin{center}
    \textbf{(二)$M=max\{X,Y\},N=min\{X,Y\}$的分布}
\end{center}

设$X,Y$是两个相互独立的随机变量,它们的分布函数分别为
$F_X(x)$和$F_Y(y)$,记$M,N$的分布函数分别为
$F_{max}(z)$和$F_{min(z)}$,则
$$F_{max}(z)=P(M\leq z)=P(X\leq z,Y\leq z)=P(X\leq z)P(Y\leq z)$$
即
$$F_{max}(z)=F_X(z)F_Y(z)$$
\begin{align}
    F_{min}(z)&=P(N\leq z)=1-P(N>z)=1-P(X>z,Y>z)\notag
    \\&=1-P(X>z)P(Y>z)\notag
\end{align}
即
$$F_{min}(z)=1-(1-F_X(z))(1-F_Y(z))$$

推广到n元随机变量:


$max\geq z \iff $所有都大于等于$z$

$min\leq z \iff $用$1-$转化为$max$

设$M=max(X_1,X_2,...,X_n)$,则
$$F_M(z)=P(M\leq z)=P(X_1\leq z,X_2\leq z,...,X_n\leq z)$$
若$X_1,X_2,...,X_n$相互独立:
$$=P(X_1\leq z)P(X_2\leq z)...P(X_n\leq z)=F_{X_1}(z)...F_{X_n}(z)$$
若$X_1,X_2,...,X_n$同分布:
$$=[F_{X_1}(z)]^n$$

设$N=min(X_1,X_2,...,X_n)$,则
$$1-F_N(z)=1-P(N\leq z)=P(N>z)=P(X_1>z,X_2>z,...,X_n>z)$$
若$X_1,X_2,...,X_n$独立,
$$=P(X_1>z)...P(X_n>z)=(1-F_{X_1}(z))...(1-F_{X_n}(z))$$
若$X_1,...,X_n$同分布,
$$=[1-F_{X_1}(z)]^2$$
\section{二元随机变量函数的分布}

\chapter{随机变量的数字特征}
\section{数学期望}
\begin{center}
    \textbf{(一)数学期望定义}
\end{center}
定义:设离散型随机变量$X$的分布律为:
$P(X=x_k)=p_k,k=1,2,...$
若级数$\sum\limits_{k=1}^{+\infty}|x_k|p_k<\infty$,则
称级数$\sum\limits_{k=1}^{+\infty}x_kp_k$的
值为$X$的数学期望,记为$E(X)$,即
$$E(X)=\sum_{k=1}^{+\infty}x_kp_k$$
~\\

定义:设连续型随机变量$X$的概率密度函数为$f(x)$,若积分
$\int_{-\infty}^{+\infty}|x|f(x)dx<\infty$,则称
积分$\int_{-\infty}^{+\infty}xf(x)dx$的值为
$X$的数学期望,记为$E(X)$,即
$$E(X)=\int_{-\infty}^{+\infty}xf(x)dx$$
数学期望简称期望,又称均值
~\\

例1.4:设$X\sim P(\lambda)$,$E(X)=\lambda$

例1.5:设$X$服从参数为$\lambda(\lambda >0)$的指数分布,
则$E(X)=\dfrac{1}{\lambda}$
\section{方差}
\section{协方差、相关系数}
\section{其它数字特征}

\end{document}