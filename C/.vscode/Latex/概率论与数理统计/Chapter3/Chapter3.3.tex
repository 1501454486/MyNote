\textbf{(一)联合概率密度函数}


定义:对于二元随机变量$(X,Y)$的分布函数$F(x,y)$,如果存在非负函数
$f$使对于任意$x$,$y$,有$$F(x,y)=\int _{-\infty}^y\int _{-\infty}^x
f(u,v)dudv$$
称$(X,Y)$为二元连续型随机变量,称$f(x,y)$为二元随机变量$(X,Y)$的
(联合)概率密度函数


\textbf{联合密度函数性质}
\begin{itemize}
    \item [1.]$f(x,y)\geq 0$
    \item [2.]$\int _{-\infty}^{+\infty}\int _{-\infty}^{+\infty}
    f(x,y)dxdy=1$
    \item [3.]设$G$是平面上区域,$(X,Y)$落在$G$内的概率
    $P\{(X,Y)\in G\}=\iint \limits_{G}f(x,y)dxdy$
    \item [4.]在$f(x,y)$的连续点$(x,y)$,有
    $\dfrac{\partial ^2 F(x,y)}{\partial x \partial y}=f(x,y)$
\end{itemize}


\textbf{(二)边际(边缘)概率密度函数}


设连续性随机变量$(X,Y)$的密度函数为$f(x,y)$,则$X,Y$的边际概率密度函数为别为:
$$f_X(x)=\int _{-\infty}^{+\infty}f(x,y)dy$$
$$f_Y(y)=\int _{-\infty}^{+\infty}f(x,y)dx$$


\textbf{(三)条件分布函数}


定义:若$P(Y=y)>0$,则在$\{Y=y\}$条件下,$X$的条件分布函数为:
$$F_{X|Y}(x|y)=P(X\leq x|Y=y)=\dfrac{P(X\leq x,Y=y)}{P(Y=y)}$$
若$P(Y=y)=0$,但对任给$\epsilon >0,P(y<Y+\epsilon \leq y+\epsilon)>0$,
则在$\{Y=y\}$条件下,$X$的条件分布函数为:
$$F_{X|Y}(x|y)=\lim\limits_{\epsilon \to 0^+}P(X\leq x|y<Y\leq y+\epsilon)
=\lim\limits_{\epsilon \to 0^+} \dfrac{P(X\leq x,y<Y\leq y+\epsilon)}{P(y<Y\leq y+\epsilon)}$$
仍记为$P(X\leq x|Y=y)$


事实上,
$$F_X(x)=F(x,+\infty)=\int _{-\infty}^x [\int _{-\infty}^{+\infty}f(u,v)dv]du
=\int _{-\infty}^xf_X(u)du$$


同理:
$$F_Y(y)=F(+\infty,y)=\int _{-\infty}^y[\int _{-\infty}^{+\infty}f(u,v)du]dv
=\int _{-\infty}^y f_Y(v)dv$$


定义:条件概率密度函数


设二元随机变量$(X,Y)$的密度函数为$f(x,y)$,$X,Y$的边际密度函数为$f_X(x),f_Y(y)$,
则在$\{Y=y\}$条件下$X$的条件密度函数为:
$$f_{X|Y}(x|y)=\dfrac{f(x,y)}{f_Y(y)},f_Y(y)>0$$


在$\{X=x\}$条件下,$Y$的条件密度函数为:
$$f_{Y|X}(y|x)=\dfrac{f(x,y)}{f_X(x)},f_X(x)>0$$


即$F_{X|Y}(x|y)=\int _{-\infty}^x\dfrac{f(u,y)}{f_Y(y)}du$
\begin{align}
    \because F_{X|Y}(x|y)&=\lim \limits_{\Delta y \to 0^+} 
    \dfrac{P(X\leq x,y<Y\leq y+\Delta y)}{P(y<Y\leq y+\Delta y)}\notag
    \\&= \lim \limits_{\Delta y \to 0^+} 
    \dfrac{\dfrac{1}{\Delta y}\int _{-\infty}^x ds \int _y^{y+\Delta y}f(u,v)dv}{\dfrac{1}{\Delta y}\int _y^{y+\Delta y}f_Y(t)dt}\notag
    \\&=\dfrac{\int _{-\infty}^xf(u,y)du}{f_Y(y)}\notag
    \\&=\int _{-\infty}^x \dfrac{f(u,y)}{f_Y(y)}du\notag
    \\\therefore F_{X|Y}(x|y)=\int _{-\infty}^x\dfrac{f(u,y)}{f_Y(y)}du \notag
\end{align}


\textbf{(四)二元均匀分布与二元正太分布}


二元均匀分布:


若二元随机变量$(X,Y)$在二维有界区域$D$上取值,且具有概率密度函数
$$
f(x,y) = \begin{cases}
    \dfrac{1}{\text{D的面积}}, & (x,y) \in D \\
    0, & \text{其他}
\end{cases}
$$


若$D_1$是$D$的子集,则
$$P\{(X,Y)\in D_1\}=\iint \limits_{D_1}f(x,y)dxdy
=\dfrac{D_1\text{的面积}}{D\text{的面积}}$$


二元正太分布:


设二元随机变量$(X,Y)$的概率密度函数为:
$$f(x,y)=\dfrac{1}{2\pi \sigma_1\sigma_2 \sqrt{1-\rho^2}}
\cdot exp\{\dfrac{-1}{2(1-\rho^2)}[\dfrac{(x-\mu_1)^2}{\sigma_1^2}
-2\rho\dfrac{(x-\mu_1)(y-\mu_2)}{\sigma_1\sigma_2}+
\dfrac{(y-\mu_2)^2}{\sigma_2^2}]\}$$


$(-\infty<x<+\infty,-\infty<y<+\infty)$


其中$\mu_1,\mu_2,\sigma_1,\sigma_2$都是常数,
且$\sigma_1>0,\sigma_2>0,-1<\rho<1$


称$(X,Y)$为服从参数为$\mu_1,\mu_2,\sigma_1,\sigma_2,\rho$
的二元正太分布,记为:
$$(X,Y)\sim N(\mu_1,\mu_2,\sigma_1^2,\sigma_2^2,\rho)$$


\begin{align}
    f_{Y|X}(y|x)&=\dfrac{f(x,y)}{f_X(x)}\notag
    \\&=\dfrac{1}{\sqrt{2\pi}\sigma_2\sqrt{1-\rho^2}}
    \cdot exp\{\dfrac{-1}{2(1-\rho^2)\sigma_2^2}[y-(\mu_2+\rho\dfrac{\sigma_2}{\sigma_1}(x-\mu_1))]^2\}\notag
\end{align}


即在$\{X=x\}$条件下,$Y$的条件分布是正太分布
$$N(\mu_2+\rho\dfrac{\sigma_2}{\sigma_1}(x-\mu_1),(1-\rho^2)\sigma_2^2)$$


同理,在$\{Y=y\}$条件下,$X$的条件分布是正太分布
$$N(\mu_1+\rho\dfrac{\sigma_1}{\sigma_2}(y-\mu_2),(1-\rho^2)\sigma_1^2)$$


\newpage