定义:设$F(x,y)$及$F_X(x),F_Y(y)$分别是随机变量
$(X,Y)$的联合分布函数及边际分布函数,
若对所有实数$x,y$有
$$P(X\leq x,Y\leq y)=P(X\leq x)P(Y\leq y)$$

即:
$$F(x,y)=F_X(x)F_Y(y)$$


称随机变量$X,Y$相互独立


若$(X,Y)$是离散型随机变量,则$X,Y$相互独立的条件
等价于:$P(X=x_i,Y=y_j)$,即:
$p_{ij}=p_{i\cdot}p_{\cdot j}$对一切$i,j$都成立.


若$(X,Y)$是连续型随机变量,$f(x,y),f_X(x),f_Y(y)$
分别是$(X,Y)$的联合密度函数和边缘密度函数,则$X,Y$
相互独立的条件等价于:$f(x,y)=f_X(x)f_Y(y)$几乎处处成立;
即平面上除去零“面积”集外,处处成立

$X$、$Y$独立$\iff\forall x,y,F(x,y)=F_X(x)F_Y(y)$

$X$、$Y$离散:$\iff P(X=x_i,Y=y_j)=P(X=x_i)P(Y=y_j)$

$\iff P(Y=y_j|X=x_i)=P(Y=y_j)$

$X$、$Y$连续:$\iff f(x,y)=f_X(x)f_Y(y)$

$\iff f_{Y|X}(y|x)=f_Y(y),\forall x,y$
~\\

定理3.4.1:连续型随机变量$X,Y$相互独立的充分必要条件是:
$$f(x,y)=m(x)\cdot n(y),|x|<+\infty,|y|<+\infty$$

即:\begin{itemize}
    \item [1.]$f$可以拆成分别关于$x,y$的两个函数的积
    \item [2.]$x,y$积分区域没有关联
\end{itemize}
~\\

例4.5 证明:对于二维正态随机变量$(X,Y)$,$X$与$Y$相互独立的充要条件是参数$\rho=0$

因为$(X,Y)$的概率密度函数为:
$$f(x,y)=\dfrac{1}{2\pi\sigma_1\sigma_2\sqrt{1-\rho^2}}\cdot
exp\{\dfrac{-1}{2(1-\rho^2)}[\dfrac{(x-\mu_1)^2}{\sigma_1^2}-
2\rho\dfrac{(x-\mu_1)(y-\mu_2)}{\sigma_1\sigma_2}+
\dfrac{(y-\mu_2)^2}{\sigma_2^2}]\}$$
又其边际密度函数的乘积为:
$$f_X(x)f_Y(y)=\dfrac{1}{2\pi\sigma_1\sigma_2}\cdot
exp\{-\dfrac{1}{2}[\dfrac{(x-\mu_1)^2}{\sigma_1^2}+
\dfrac{(y-\mu_2)^2}{\sigma_2^2}]\}$$

$\Leftarrow$:如果$\rho=0$,则对$\forall x,y$,有
$f(x,y)=f_X(x)f_Y(y)$,即$X,Y$相互独立

$\Rightarrow$:反之,若$X,Y$相互独立,由于$f(x,y),f_X(x),f_Y(y)$都是连续函数,

故对$\forall x,y,$有$f(x,y)=f_X(x)f_Y(y)$

特别的,有$f(\mu_1,\mu_2)=f_X(\mu_1)f_Y(\mu_2),$

即:$\dfrac{1}{2\pi\sigma_1\sigma_2\sqrt{1-\rho^2}}=\dfrac{1}{2\pi\sigma_1\sigma_2}$


\begin{center}
    \textbf{一般n元随机变量的一些概念和结果}
\end{center}

n元随机变量定义:

设$E$是一个随机试验,它的样本空间是$S=\{e\};$
设$X_1=X_1(e),X_2=X_2(e),...,X_n=X_n(e)$是定义在$S$上的随机变量,
由它们构成的一个$n$元向量$(X_1,X_2,...X_n)$称为$n$元随机变量
~\\

n元随机变量分布函数:

对于任意$n$个实数$x_1,x_2,...,x_n,$n元函数:

$F(x_1,x_2,...,x_n)=P(X_1\leq x_1,X_2\leq x_2,...,X_n\leq x_n)$

称为n元随机变量$(X_1,X_2,...,X_n)$的分布函数
~\\

离散型随机变量的分布律、连续型随机变量的概率密度函数与之前的定义相同。
~\\

\begin{center}
    \textbf{多元随机变量相互独立:}
\end{center}


若对于所有的$x_1,x_2,...,x_n,$有:
$$F(x_1,x_2,...,x_n)=F_{X_1}(x_1)F_{X_2}(x_2)...F_{X_n}(x_n)$$

则称$X_1,X_2,...,X_n$是相互独立的
~\\

$(X_1,X_2,...,X_m)$与$(Y_1,Y_2,...,Y_n)$的独立性


设$(X_1,X_2,...,X_m)$的分布函数为$F_1(x_1,x_2,...,x_m),$
$(Y_1,Y_2,...,Y_n)$的分布函数为$F_2(y_1,y_2,...,y_n),$
$(X_1,X_2,...,X_n,Y_1,Y_2,...,Y_n)$的分布函数为:
$$F(x_1,x_2,...,x_m,y_1,y_2,...,y_n)$$

若$F(x_1,x_2,...,x_m,y_1,y_2,...,y_n)=F_1(x_1,x_2,...,x_n)F_2(y_1,y_2,...,y_n),$

称$(X_1,X_2,...,X_m)$与$(Y_1,Y_2,...,Y_n)$相互独立
~\\

定理:设$(X_1,X_2,...,X_m)$与$(Y_1,Y_2,...,Y_n)$相互独立,
则$X_i(i=1,2,...,m)$与$Y_j(j=1,2,...,n)$相互独立


若$h(x_1,x_2,...,x_m)$与$g(y_1,y_2,...,y_n)$是连续函数,
则$h(X_1,X_2,...,X_m)$和$g(Y_1,Y_2,...,Y_n)$相互独立
~\\

\begin{center}
    \textbf{(一)$Z=X+Y$的分布}
\end{center}
设$(X,Y)$为离散型随机变量,分布律为$P(X=x_i,Y=y_j)=p_{ij},i,j=1,2,...$
设$Z$的可能取值为$z_1,z_2,...,z_k,...,$则$Z=X+Y$的分布律为
$$P(Z=z_k)=P(X+Y=z_k)=\sum\limits_{i=1}^{+\infty}P(X=x_i,Y=z_k-x_i),k=1,2,...$$
或$P(Z=z_k)=\sum\limits_{j=1}^{+\infty}P(X=z_k-y_j,Y=y_j),k=1,2,...$

特别的,当$X$与$Y$相互独立时,
$$P(Z=z_k)=\sum_{i=1}^{+\infty}P(X=x_i)P(Y=z_k-x_i),k=1,2,...$$
或
$$P(Z=z_k)=\sum_{j=1}^{+\infty}P(X=z_k-y_j)P(Y=y_j),k=1,2,...$$
~\\


设连续型随机变量$(X,Y)$的密度函数为$f(x,y)$,则$Z=X+Y$
的分布函数为:
\begin{align}
    F_Z(z)&=P(Z\leq z)=\iint\limits_{x+y\leq z}f(x,y)dxdy\notag
    \\&=\int_{-\infty}^{+\infty}[\int_{-\infty}^{z-y}f(x,y)dx]dy\notag
    \\&=\int_{-\infty}^{+\infty}[\int_{-\infty}^zf(u-y,y)du]dy\notag
    \\&=\int_{-\infty}^z[\int_{-\infty}^{+\infty}f(u-y,y)dy]du\notag
    \\&=\int_{-\infty}^zf_Z(u)du\notag
\end{align}
固定$z,y,$令$u=x+y$

故$Z$的密度函数为:
$$f_Z(z)=\int_{-\infty}^{+\infty}f(z-y,y)dy$$

由$X,Y$的对称性:
$$f_Z(z)=\int_{-\infty}^{+\infty}f(x,z-x)dx$$

当$X$与$Y$相互独立时,
$$f_Z(z)=\int_{-\infty}^{+\infty}f_X(z-y)f_Y(y)dy=
\int_{-\infty}^{+\infty}f_X(x)f_Y(z-x)dx$$
称为卷积公式
\begin{center}
    \textbf{(二)$M=max\{X,Y\},N=min\{X,Y\}$的分布}
\end{center}

设$X,Y$是两个相互独立的随机变量,它们的分布函数分别为
$F_X(x)$和$F_Y(y)$,记$M,N$的分布函数分别为
$F_{max}(z)$和$F_{min(z)}$,则
$$F_{max}(z)=P(M\leq z)=P(X\leq z,Y\leq z)=P(X\leq z)P(Y\leq z)$$
即
$$F_{max}(z)=F_X(z)F_Y(z)$$
\begin{align}
    F_{min}(z)&=P(N\leq z)=1-P(N>z)=1-P(X>z,Y>z)\notag
    \\&=1-P(X>z)P(Y>z)\notag
\end{align}
即
$$F_{min}(z)=1-(1-F_X(z))(1-F_Y(z))$$

推广到n元随机变量:


$max\geq z \iff $所有都大于等于$z$

$min\leq z \iff $用$1-$转化为$max$

设$M=max(X_1,X_2,...,X_n)$,则
$$F_M(z)=P(M\leq z)=P(X_1\leq z,X_2\leq z,...,X_n\leq z)$$
若$X_1,X_2,...,X_n$相互独立:
$$=P(X_1\leq z)P(X_2\leq z)...P(X_n\leq z)=F_{X_1}(z)...F_{X_n}(z)$$
若$X_1,X_2,...,X_n$同分布:
$$=[F_{X_1}(z)]^n$$

设$N=min(X_1,X_2,...,X_n)$,则
$$1-F_N(z)=1-P(N\leq z)=P(N>z)=P(X_1>z,X_2>z,...,X_n>z)$$
若$X_1,X_2,...,X_n$独立,
$$=P(X_1>z)...P(X_n>z)=(1-F_{X_1}(z))...(1-F_{X_n}(z))$$
若$X_1,...,X_n$同分布,
$$=[1-F_{X_1}(z)]^2$$