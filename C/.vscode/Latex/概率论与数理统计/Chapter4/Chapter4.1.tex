\begin{center}
    \textbf{(一)数学期望定义}
\end{center}
定义:设离散型随机变量$X$的分布律为:
$P(X=x_k)=p_k,k=1,2,...$
若级数$\sum\limits_{k=1}^{+\infty}|x_k|p_k<\infty$,则
称级数$\sum\limits_{k=1}^{+\infty}x_kp_k$的
值为$X$的数学期望,记为$E(X)$,即
$$E(X)=\sum_{k=1}^{+\infty}x_kp_k$$
~\\

定义:设连续型随机变量$X$的概率密度函数为$f(x)$,若积分
$\int_{-\infty}^{+\infty}|x|f(x)dx<\infty$,则称
积分$\int_{-\infty}^{+\infty}xf(x)dx$的值为
$X$的数学期望,记为$E(X)$,即
$$E(X)=\int_{-\infty}^{+\infty}xf(x)dx$$
数学期望简称期望,又称均值
~\\

例1.4:设$X\sim P(\lambda)$,$E(X)=\lambda$

例1.5:设$X$服从参数为$\lambda(\lambda >0)$的指数分布,
则$E(X)=\dfrac{1}{\lambda}$